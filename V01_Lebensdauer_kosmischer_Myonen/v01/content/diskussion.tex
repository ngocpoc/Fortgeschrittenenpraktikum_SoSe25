\section{Diskussion}
\label{sec:Diskussion}
\subsection{Allgemein}
Beim Aufbau des Versuchs traten große Probleme bei der Verkabelung auf. Die Weiterleitung von Signal war oftmals gestört aus 
offensichtlichen Grund. Daher ist ein Fehler in den Messwerten aufgrund der Messelektronik nicht auszuschließen. Außerdem war 
der Umgang mit dem Computerprogramm zur Messaufzeichnung problematisch, da es bei einigen Messversuchen nichts aufzeichnete. 

\subsection{Verzögerungszeit vor der Koinzidenzschaltung}
Bereits vor der Koinzidenzschaltung sollte das Signal aus beiden Photomultipliern auf $30 \,\sfrac{\text{Pulse}}{\unit{\second}}$ gesenkt 
werden. Dies war allerdings nicht möglich, da ein Photomultiplier grundsätzlich eine viel höhere Pulsrate lieferte, als der andere. 
Daher konnte das eine Signal nur auf ungefähr $100 \,\sfrac{\text{Pulse}}{\unit{\second}}$ heruntergeregelt werden und das andere 
auf $30 \,\sfrac{\text{Pulse}}{\unit{\second}}$. Dies könnte zu Fehlern in den Messdaten führen. Außerdem ist in den für 
Verzögerungszeit aufgenommene Messwerte kein klares Plateau zu erkennen, was eine Folge der unterschiedlichen Pulsraten sein könnte. 
Demnach könnte die Einstellung der Verzögerung als $2 \, \unit{\nano\second}$ nicht ideal gewesen sein und Einfluss auf den Rest des 
Versuchs gehabt haben. 

\subsection{Kalibrierung des MCAs}
Die Messwerte der Kalibrierung des MCAs liegen allesamt augenscheinlich sehr nah an der Ausgleichsgerade, was für eine hohe 
Genauigkeit spricht. Diese Messwerte konnten nicht durch die Photomultiplier verfälscht werden, da die Pulse für die Messung 
vom Doppelpulsgenerator erzeugt wurden. Dies kann ein Grund für die Genauigkeit sein. Zusätzlich erfolgte die Messung computergestützt,
was den Einfluss der menschlichen Reaktionszeit nebensächlich macht. 

\subsection{Bestimmung der Lebendauer von Myonen und der Untergrundrate}
Die Problematik, die sich bei der Langzeitmessung ergab, war, dass das Computerprogramm die Messwerte nicht aufgezeichnet hatte, als
die Messung nach $2$ Tagen gestoppt werden sollte. Allerdings konnten die Daten dann doch nach $4$ Tagen auf dem Computer von einer 
anderen Gruppe gefunden werden, die den Versuch an diesem Tag durchführen sollte. Aus diesem Grund ist die genaue Messdauer nicht
bekannt und die Anzahl der Start- und Stoppsignale ebenso. Daher konnte die Untergrundrate nur durch die Ausgleichsfunktion und 
mit einer theoretischen Einfallrate bestimmt werden. Die Abweichung zwischen 
$$U_{\text{theo}} = 3,5 \,\, \frac{\text{Counts}}{\text{Kanal}}\,\, \text{und} \,\, U = (0,50 \pm 0,29) \,\, \frac{\text{Counts}}{\text{Kanal}}$$
liegt bei $85,71 \, \%$. Die Abweichung zwischen der theoretischen Lebensdauer $2,2 \, \unit{\mu\second}$ und der gemessenen Lebensdauer 
$(2,0 \pm 0,1)\, \unit{\mu\second}$ beträgt $9,09 \, \%$. 

