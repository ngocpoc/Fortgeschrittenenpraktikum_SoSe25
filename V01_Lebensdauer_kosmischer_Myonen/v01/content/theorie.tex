\section{Zielsetzung}
\label{sec:Zielsetzung}
Ziel dieses Versuchs ist die Bestimmung der Lebensdauer von Myonen aus einer Messreihe von Indiviuallebensdauern einzelner Myonen.
\section{Theorie}
\label{sec:Theorie}
\subsection{Standardmodell der Elementarteilchen}
Das Standardmodell der Teilchenphysik, wie es z.B. in \cite{Astroteilchenphysik} erklärt wird, unterteilt die Elementarteilchen in Quarks, Leptonen und Eichbosonen.
Quarks und Leptonen gehören zur Gruppe der Fermionen und lassen sich jeweils in drei Generationen einteilen, 
die sich insbesondere in ihrer Masse und Lebensdauer unterscheiden.\\
% Quarks tragen eine sogenannte Farbladung und nehmen an der starken Wechselwirkung teil. 
Leptonen unterliegen der schwachen Wechselwirkung. 
Zu den geladenen Leptonen zählen das Elektron, das Myon und das Tauon, die jeweils eine elektrische Ladung von $-\symup{e}$ besitzen und zudem der elektromagnetischen Wechselwirkung 
unterliegen. 
Die zugehörigen Neutrinos ($\nu_{\symup{e}},\, \nu_\mu,\, \nu_\tau$) sind elektrisch neutral. Im Gegensatz zu Elektronen sind Myonen und Tauonen instabil und zerfallen mit charakteristischen Lebensdauern $\tau$.

\subsubsection{Myonen}
\label{sec:Myonen}
Die folgenden Ausführungen zur Entstehung, Lebensdauer und den Zerfällen von Myonen basieren im Wesentlichen auf \cite{Astroteilchenphysik}.
Myonen entstehen in der Erdatmosphäre in Höhen von etwa $15{-}20\,\unit{\kilo\meter}$ durch den Zerfall von Pionen. 
Diese Pionen entstehen zuvor bei der Wechselwirkung hochenergetischer kosmischer Protonen mit den Atomkernen der Luftmoleküle. 
Aufgrund ihrer kurzen Lebensdauer zerfallen die Pionen wie folgt:
\begin{align*}
	\pi^+ &\rightarrow \mu^+ + \nu_\mu \\
	\pi^- &\rightarrow \mu^- + \bar{\nu}_\mu\,.
\end{align*}

Die dabei entstehenden Myonen besitzen eine mittlere Lebensdauer von $2,2\,\unit{\micro\second}$ \cite{PDG2024} und zerfallen in Elektronen, Neutrinos und Antineutrinos:
\begin{align*}
	\mu^- &\rightarrow \text{e}^- + \bar{\nu}_\text{e} + \nu_\mu \\
	\mu^+ &\rightarrow \text{e}^+ + \nu_\text{e} + \bar{\nu}_\mu\,.
\end{align*}

Da sich Myonen mit Geschwindigkeiten nahe der Lichtgeschwindigkeit bewegen, erreichen sie trotz ihrer kurzen Lebensdauer die Erdoberfläche und können sogar Tiefen von bis zu $160\,\unit{\meter}$ Gestein (etwa $400\,\unit{\meter}$ Wasseräquivalent) durchdringen, wie in \cite{Astroteilchenphysik} beschrieben.

\subsection{Mittlere Lebensdauer von Elementarteilchen}
Die Herleitung der mittleren Lebensdauer $\tau$ wird ausführlich in \cite{Techniques} behandelt. Die Zerfallsgeschwindigkeit eines Teilchens 
ist proportional zur Anzahl der verbleibenden Teilchen $N(t)$ zu einem bestimmten Zeitpunkt $t$. Diese Beziehung wird durch die folgende 
Differentialgleichung beschrieben:
\begin{align}
\symup{d}N = -\lambda N(t)\,\symup{d}t \,, % \label{eqn:dN}
\end{align}
wobei $\lambda$ die Zerfallskonstante ist, die die Wahrscheinlichkeit für den Zerfall eines Teilchens pro Zeiteinheit beschreibt. 
Die Lösung dieser Gleichung ergibt das exponentielle Zerfallsgesetz:
\begin{align}
N(t) = N_0 \symup{e}^{-\lambda t} \,, \label{eqn:Zerfallsgesetz}
\end{align}
wobei $N_0$ die Anfangszahl der Teilchen darstellt. Die mittlere Lebensdauer $\tau$ ist definiert als die Zeit, nach der die Zahl der 
verbleibenden Teilchen auf den Wert $\sfrac{1}{\symup{e}}$ des ursprünglichen Werts abgefallen ist. Sie ist mit der Zerfallskonstanten $\lambda$ durch die Beziehung
\begin{align}
\tau = \frac{1}{\lambda}\,, \label{eqn:tau}
\end{align}
verknüpft.



%$7153920 - 7153920 * 0,00041383 = 7150959$
%und eine Fehlerhafte Stoppanzahl von $7150959 \cdot P(1) = 2959$. Diese Verteilen sich auf die ersten 412 Kanäle. Daher kommt es 
%zu der Untergrundrate pro Kanal von $$U_{\text{theo}} = 7 \,\, \frac{\text{Counts}}{\text{Kanal}}$$.
%
%Bei einer Langzeitmessung von 2 Tagen ergibt sich 
%
%eine Untergrundrate pro Kanal von $$U_{\text{theo}} = 3,5 \,\, \frac{\text{Counts}}{\text{Kanal}}$$.

%Bei einer Langzeitmessung von $\Delta t_{\text{L}} = 4 Tage $  ergibt sich eine durchschnittliche Startanzahl von $7153920 - 7153920 * 0,00041383 = 7150959$
%und eine Fehlerhafte Stoppanzahl von $7150959 \cdot P(1) = 2959$. Diese Verteilen sich auf die ersten 412 Kanäle. Daher kommt es 
%zu der Untergrundrate pro Kanal von $$U_{\text{theo}} = 7 \,\, \frac{\text{Counts}}{\text{Kanal}}$$.


% \subsection{Vorbereitungsaufgaben}
% \label{sec:Vorbereitungsaufgaben}\