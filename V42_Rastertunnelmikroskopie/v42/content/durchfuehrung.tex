\section{Durchführung}
\label{sec:Durchführung}
Für die Messung wird ein Rastetunnelmikroskop mit einer Spitze aus PtIr-Draht verwendet. 
Vor Beginn der Messung werden das STM, der Draht sowie das verwendete Werkzeug zur Herstellung der Spitze gründlich gereinigt und desinfiziert, um Störungen durch Schmutz oder Staub zu vermeiden. Zusätzlich werden Handschuhe während der Durchführung getrage, um ebenfalls Schmutz und Staub zu verhindern.\\

Die STM-Spitze wird hergestellt, indem der Draht mit einer Zange fixiert und mit einem Seitenschneider in einem Winkel von ca. $\SI{45}{\degree}$ gerissen wird. 
Dabei wird der Seitenschneider gleichzeitig zugedrückt und vom Draht weggezogen. 
Diese Spitze wird in die dafür vorgesehene Vertiefung unter dem Bügel des STMs mithilfe einer Pinzette eingsetzt. \\

Die Probe wird mithilfe eines Magneten am Messkopf befestigt. 
Der Messkopf wird mit der Probe an die Spitze herangeführt, wobei darauf zu achten ist, dass die Spitze die Probe nicht berührt, um die Spitze nicht zu beschädigen\\

Um Luftströmungen während der Messung zu minimiern, wird eine Abdeckung über das Rastertunnelmikroskop gesetzt. Mithilfe der Messsoftware wird die Probe näher an die Spitze positioniert. Durch den Befehl „Approach“ startet die automatische Annäherung, bis ein Tunnelstrom detektiert wird. Die Messung beginnt anschließend automatisch.

Für die Messung der HOPG Probe wird eine Bildgröße zwischen $2$ und $\SI{5}{\nano\meter}$ und eine Scan-Geschwindigkeit $v_{\text{scan}}$ von $\SI{0.05}-\SI{0.15}{\second\per\text{line}}$ eingestellt. Bei der Vermessung der Gold Probe eine Bildgröße von ca. $\SI{300}-\SI{500}{\nano\meter}$ und eine langsamere Scan-Geschwindigkeit im Vergleich zur HOPG Probe gewählt. Um die STM-Abbildungen zu optimieren, werden die Werte des PID-Reglers angepasst. 