\section{Diskussion}
\label{sec:Diskussion}
Das verwendete Rastertunnelmikroskop ist unter anderem anfällig für die in der Theorie beschriebenen Faktoren der Piezokristalle wie Nichtlinearität, Hysterese, Creep, Cross Coupling sowie Alterung, verfügt allerdings nicht über Hardwarekorrekturen, um diese ausgleichen zu können. Das Gerät verfügt allein über eine Softwarekorrektur. Daher ist anzunehmen, dass die experimentell bestimmten Werte diese Unsicherheiten widerspiegeln könnten. Außerdem stellte die Herstellung des Spitzen eine weitere Schwierigkeit dar. Um konsistent gut funktionierende Spitzen herzustellen, bedarf es an Übung, die während des Experimentes nur unzureichend erworben werden konnte. Qualitativ schlechte Spitzen könnten für Unterschiede in den vorwärts und rückwärts Scans sorgen und verschlechtern die Qualität der Scans, sodass die einzelnen Höhen und Tiefen schlechter abgelesen werden können. 

\subsection{Abweichung der Messung von HOPG vom Literaturwert}
Der Literaturwert des Übernächste-Nachbarn-Abstand beträgt $d_{\text{Literatur}} = 0,246 \, \unit{\nano\meter}$. Die Abweichung zum experimentell berechneten Wert $d_{\text{exp}} = (0,223 \pm 0,001)  \, \unit{\nano\meter}$ beträgt $5,28 \, \%$. Auffallend ist der Unterschied der Länge der beiden Gittervektoren. Ausgehend von der Gitterstruktur sollten diese annähernd die gleiche Länge besitzen. Der Winkel zwischen den beiden Gittervektoren wurde experimentell zu $\alpha_{\text{exp}} = (49,37 \pm 0,39)°$ bestimmt. Zum Literaturwert $\alpha_{\text{Lit}} = 60°$ besteht eine Abweichung von $17,72 \, \%$. Die Ursache für die Abweichung des Winkels und des Übernächste-Nachbarn-Abstand könnte eine zur Spitze verkippte Probe sein, die für eine Verzerrung in eine Richtung sorgt. Trotz dieses möglichen Fehlerquelle ist die Abweichung als recht gering einzustufen. 

\subsection{Beurteilung der Oberflächenmessung von Gold}
Die Oberflächenscans von Gold waren bei der Versuchsdurchführung sehr schwierig aufzunehmen, da die Spitze immer wieder in die Probe hineinfuhr und daher die Probe nicht vermessen konnte. Deswegen ist die Qualität des eigenen Scans in Frage zu stellen. Außerdem sind in den ausgewählten Höhenprofilen keine deutliche Schichten zu erkennen. Die Gitterkonstante $a$ von Gold beträgt $0,409 \, \unit{\nano\meter}$. Die Gitterkonstante ist bei einem kubisch flächenzentrierten Gitter nicht gut mit der Höhe einer Atomschicht zu vergleichen. Die Bindungslänge $l$, der Abstand zwischen zwei Atomen in einer chemischen Bindung, lässt sich besser dazu heranziehen. Bei kubisch flächenzentrierten Gittern wird $l$ mithilfe von 
$l = a \cdot \frac{\sqrt{2}}{2}$ berechnet \cite{Bindungslaenge}. $l$ ergibt sich bei Gold zu $0,288 \, \unit{\nano\meter}$. Der experiementell gemessene Wert ist $\Delta z = (0,259 \pm 0,021)\,  \unit{\nano\meter}$. Die Abweichung zwischen dem experimentell bestimmten Wert und der Bindungslänge beträgt $10,07 \, \%$. Diese Abweichung ist für die möglichen Fehlerquellen als gering einzuordnen.
%288,37 pm