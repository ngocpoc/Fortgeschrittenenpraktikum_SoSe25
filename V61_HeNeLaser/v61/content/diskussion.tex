\section{Diskussion}
\label{sec:Diskussion}

\subsection{Stabilitätsbedingung und das Frequenzspektrum}
Die theoretisch berechnete maximal mögliche Resonatorlänge von $2{,}8 \, \unit{\meter}$ für einen stabilen Laser konnte zwar nicht ganz erreicht werden experimentell, allerdings ließ sich auf der größt möglichen Länge der Laser immernoch stabilisieren. Die Stabilitätsbedingung konnte daher bis zu $2{,}05 \, \unit{\meter}$ bestätigt werden. 
Die Abweichung des experimentell ermittelten Wertes der Dopplerverbreiterung beträgt $1{,}56 \, \unit{\giga\hertz}$. Die Abweichung vom Theoriewert $1{,}5 \, \unit{\giga\hertz}$, der im \autoref{sec:dopplerverbreiterung} eingeführt wird, beträgt $4{,}00 \, \%$. 

\subsection{Frequenzabstände der longitudinalen Moden}
Die Steigung $m$ der Ausgleichsgerade der Frequenzdifferenzen in \autoref{fig:Multimoden} sollte nach Gleichung \ref{eqn:modenabstand} $\frac{2}{c}$ entsprechen. Die Abweichung zwischen dem Theoriewert und experimentellen Wert beträgt $0{,}61 \, \%$. 

\subsection{Moden des Lasers}
Der theoretisch erwartete Verlauf entspricht dem experimentell gemessenen Verlauf der Kurve bis auf einige Ausreißer in der Nähe der Maxima. 

\subsection{Polarisation des Lasers}
Die Polarisierung des Lasers folgt der erwarteten eines linear polarisierten Lichts wie der Laser es produziert. 

\subsection{Wellenlänge des Lasers} 
Die experimentell festgestellte Wellenlänge des Lasers ohne Kleinwinkenäherung ist $524 \, \unit{\nano\meter}$. Die Abweichung vom Theoriewert $633 \, \unit{\nano\meter}$ beträgt $17{,}21 \, \%$. Diese Abweichung ist größer als alle anderen des Versuchs. 
Gründe dafür könnten die nicht perfekte Kalibration des Lasers sein, Umgebungslicht oder ein ungenaues Gitter. Außerdem schwankte die vom Gerät angezeigte Intensität sehr stark obwohl nichts am Versuch verändert wurde. Dies könnte auch die Ausreißer in der Messung der Moden erklären. Allgemein waren die Abweichungen allerdings in einem Rahmen, der für einen Versuch im Fortgeschrittenen Praktikum nicht ungewöhnlich sind. 