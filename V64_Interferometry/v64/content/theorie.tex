\section{Objective}
\label{sec:Objective}
The goal of the experiment is to familiarize oneself with the workings of interferometry and a Sagnac-Interferometer. 
Using this knowledge, the contrast of the interferometer and the refractory index of glass and air are determined. 

\section{Theoratical Background}
\label{sec:Theorie}
The theoretical background is based on \cite{Optik} and \cite{anleitungV64}.
\subsection{Coherence, polarization and interference of light}
The coherence of a wave describes how well the phase is maintained throughout the propagation. The temporal coherence measures how long a 
wave holds the same shape periodically and the spatial coherence measures the phase unity across different points in the wavefront.
The coherence is linked to the spectral bandwidth the light source has. In general, a narrow bandwidth leads to a long coherence time.
 The spatial coherence can be expressed as 
\begin{equation*}
   K_{\text{AB}} (\tau) =  \frac{\Gamma_{\text{AB}}(\tau)}{\sqrt{\Gamma_{\text{AA}}(0) \cdot \Gamma_{\text{BB}}(0)}} 
\end{equation*}
by using
\begin{equation*}
   \Gamma_{\text{AB}}(\tau) = \langle E(r_{\text{A}},t)E^{*}(r_{\text{B}},t + \tau) \rangle \, .
\end{equation*}
The amplitudes $E$ are sampled at the different locations A and B at a time-difference of $\tau$ and compared as a function of $t$. For the timely coherence the same equation is used with $\text{A} = \text{B}$.
The degree of coherence a wave can have is ranging from 0 (fully incoherent) to 1 (fully coherent).
Another property of light is its polarization. Light can be linearly polarized, circularly or elliptically polarized or unpolarized. 
In case of a linearly polarized light the electric field osciallates in a fixed plane. If it is circularly or elliptically polarized the 
electric field rotates over time around the direction of propagation. 
For two beams to interfere with each other they both must be coherent, overlapping spatially and be polarized linearly in the same direction or have
linear compounds of polarization in the same direction. If a beam is circularly or elliptically polarized, the parallel and perpendicular components are viewed seperatly and the parallel components can interfere with each other. Therefore, two linearly polarized beams cannot interfere if the polarization planes are perpendicular to each other (because there is no component parallel to each other to make interference possible). 

\subsection{Sagnac-Interferometer}
\label{subsec:Sagnac_Interferometer}
The Sagnac-Interferometer is a type of interferometer where the initial light beam is split into two by using a polarizing beam-splitter 
cube (PBSC). The two beams propagate along the same paths just in opposite directions and are recombined afterwards. They are polarized 
perpendicular to each other because of the PBSC.

  \subsection{Contrast}
  To be able to take optimal measurements, the contrast, also called visibility, of the interferometer should be high. 
  The contrast $\nu$ is defined
   as 
  \begin{equation}
    \nu = \frac{I_{\text{max}}- I_{\text{min}}}{I_{\text{max}}+ I_{\text{min}}}\, .
    \label{eqn:contrast}
  \end{equation}
 $I_{\text{max}}$ is the maximum intensity and $I_{\text{min}}$ is the minimum intensity of the interfering beams. The ideal contrast is therefore equal to $1$ because
 $I_{\text{min}}$ is equal to $0$ in this case. The maximal and minimal intensity is dependent on the electric field of the beams taking 
 part in the interference. The relation between the electric field $E$ and the corresponding intensity $I$ is $I = \langle |E|² \rangle$. \\
 In case of the Sagnac-Interferometer the initially linearly polarized laser beam is split into two beams by the PBSC. The measured 
 intensity is therefore dependent on the two beams and their electric field. The electric fields oscillate with a frequency $\omega$. 
 At the point where the two beams are recombined, the waves have a phase difference in comparison to each other. This phase difference
 is labeled $\delta$. Therefore the dependency of the intensity can be expressed through
 \begin{equation*}
    I \propto \langle |E_1 \cos(\omega t) + E_2 \cos(\omega t + \delta)|²\rangle \, .
 \end{equation*}
 The maximal intensity is measured when $\delta$ is $\delta_{\text{max}} = 0$ and the minimal intensity is
 measured when $\delta$ is $\delta_{\text{min}} = \pi$. 
 %cos(x + \pi) = - cos(x)
 This leads to the maximal and minimal intensity being 
 \begin{align*}
    I_{\text{max/min}} &\propto \langle |(E_1 \pm E_2) \cos(\omega t)|²\rangle \\
    &= |(E_1 \pm E_2)|² \langle |\cos(\omega t)|²\rangle \\
    &= \frac{1}{2} |(E_1 \pm E_2)|² \, .
 \end{align*}
 The two beams have a linear polarization perpendicular to each other after passing through the PBSC. 
 Therefore the amplitudes $E_1$ and $E_2$ of the beams can be expressed as 
 \begin{equation}
    E_1 = E_0 \cdot \cos(\phi) \,\, \text{and} \,\, E_2 = E_0 \cdot \sin(\phi) \, .
    \label{eqn:E_Felder}
 \end{equation}
 $E_0$ is the amplitude of the electric field of the initial beam and $\phi$ is the angle between the initial beam and the PBSC. 
 The electric field of the initial beam is assumed to be $E = E_0 \cdot \cos(\omega t)$. Therefore the initial intensity is
 $I_{\text{Laser}} = \langle |E_0 \cdot \cos(\omega t)|² \rangle = \frac{1}{2} |E_0|²$.
 The intensity can be expressed as 
 \begin{align*}
    I_{\text{max/min}} &\propto \frac{1}{2} \, |E_0 \cdot \cos(\phi) \pm E_0 \cdot \sin(\phi)|² \\
    &= \frac{1}{2} \,|E_0|² \cdot |\cos(\phi) \pm \sin(\phi)|² \\
    &= I_{\text{Laser}}  \cdot |\cos²(\phi) \pm 2\sin(\phi)\cos(\phi) + \sin²(\phi)| \\
    &= I_{\text{Laser}}  \cdot |1 \pm 2\sin(\phi)\cos(\phi)|
 \end{align*}
 by using the established definitions of $E_1$ and $E_2$. 
 These intensities can be used to calculate the contrast of the Sagnac-Interferometer as following 
 \begin{align}
    \nu(\phi) &= \frac{I_{\text{Laser}}  \cdot |1 + 2\sin(\phi)\cos(\phi)|- I_{\text{Laser}}  \cdot |1 - 2\sin(\phi)\cos(\phi)|}{I_{\text{Laser}}  \cdot |1 + 2\sin(\phi)\cos(\phi)|+ I_{\text{Laser}}  \cdot |1 - 2\sin(\phi)\cos(\phi)|} \nonumber \\
    &= \frac{|4 \sin(\phi)\cos(\phi)|}{2} = 2 \, |\sin(\phi)\cos(\phi)| \, . 
    \label{eqn:contrast_Sagnac}
 \end{align}

 \subsection{Refractive index}
 Light waves travel at a different velocity $v_{\text{medium}}$ dependent on the material they propagate through. 
 The refractive index $n$ is defined as 
 \begin{equation}
    n = \frac{\symup{c}}{v_{\text{medium}}} \label{n_generell} \, .
 \end{equation}
 $\symup{c}$ is the velocity of light in vacuum. 
 The refractive index of gases can be measured by using an interferometer because a changing refractive 
 index $\Delta n$ leads to the wave acquiring the additional phase 
 \begin{equation}
    \delta_{\text{gas}} = \frac{2 \pi}{\lambda_{\text{vac}}} \Delta n L \, . \label{eqn:delta_gas}
 \end{equation}
 $\lambda_{\text{vac}}$ is the wavelength of light in vacuum and $L$ is the length of the gas cell. 
 The refractive index of glass can be measured through measuring the number of interference maxima $M$
 because it relates to the phase difference as following 
 \begin{equation}
    M = \frac{\delta_{\text{glass}}}{2\pi} \label{eqn:M} \, .
 \end{equation}
 The phase shift $\delta_{\text{glass}}$ can be expressed as 
 \begin{equation}
    \delta_{\text{glass}}(\Theta) = \frac{2 \pi}{\lambda_{\text{vac}}} d \left( \frac{n_{\text{glass}}-1}{2n_{\text{glass}}} \Theta² + \mathcal{O}(\Theta⁴)\right) 
    \label{eqn:delta_glass_vontheta}
 \end{equation}
 for small angles of rotation. 
 $d$ is the thickness of the glass plate and $n_{\text{glass}}$ is the refractive index of the glass. 
 Because both beams of light propagate along the same path in different directions the phase difference between the two waves is 
 \begin{equation}
    \delta_{\text{diff}} = \delta(\Theta + \Theta_0) - \delta(\Theta - \Theta_0) \label{eqn:delta_diff} \, .
 \end{equation}
 $\Theta_0$ is the tilt that both glass panels have at the beginning. 
 %The refractive index can be calculated using 
 %the formulas (\ref{eqn:M}), (\ref{eqn:delta_glass_vontheta}) and (\ref{eqn:delta_diff}) as following 
 %\begin{align*}
 %   M &= \frac{d}{\lambda_{\text{vac}}} \frac{n_{\text{glass}}-1}{2n_{\text{glass}}} \left( (\Theta + \Theta_0)² - (\Theta - \Theta_0)²\right) \\
 %   &= \frac{d}{\lambda_{\text{vac}}} \frac{n_{\text{glass}}-1}{2n_{\text{glass}}} \Theta \Theta_0    
 %   \, . 
 %\end{align*}
 %Solving for $n_{\text{glass}}$ yields 
 %\begin{equation}
 %   n_{\text{glass}} = \frac{2d \Theta \Theta_0}{2d \Theta \Theta_0 - \lambda_{\text{vac}}M} \label{eqn:n_glass_umgestellt}\, .
 %\end{equation}
 The refractive index is also dependent on the polarizability $\alpha$, the temperature $T$ and the pressure $p$ which is shown in the 
 Lorentz-Lorenz equation 
 \begin{equation}
   \frac{n²-1}{n²+2} = \frac{\alpha p}{3 \varepsilon_0 k_b T} 
   \label{eqn:Lorentz_Lorenz}
 \end{equation}