\section{Analysis}
\label{sec:Analysis}

\subsection{Contrast of the interferometer}
\label{sec:Contrast}
\begin{table}[h]
    \centering
    \caption{Measured voltages $U_{\text{max/min}}$ and their arithmetic means $\bar{U}_{\text{max/min}}$ at various polarization angles $\phi$ used to determine the interferometer contrast.}
    \label{tab:contrast}
    \begin{tblr}{colspec={c| c c c |c c c| c c }}
        \toprule
        $\phi\,[\unit{\degree}]$ & $U_{\text{max}}^{(1)}\,[\unit{\volt}]$ & $U_{\text{max}}^{(2)}\,[\unit{\volt}]$ & $U_{\text{max}}^{(3)}\,[\unit{\volt}]$ & $U_{\text{min}}^{(1)}\,[\unit{\volt}]$ & $U_{\text{min}}^{(2)}\,[\unit{\volt}]$ & $U_{\text{min}}^{(3)}\,[\unit{\volt}]$ & $\bar{U}_{\text{max}}\,[\unit{\volt}]$ & $\bar{U}_{\text{min}}\,[\unit{\volt}]$ \\
        \midrule
        0   & 1.69 & 1.58 & 1.57 & 1.38 & 1.41 & 1.43 & $1.61\pm0.07$ & $1.41\pm0.03$ \\
        10  & 1.35 & 1.36 & 1.37 & 0.89 & 0.90 & 0.91 & $1.36\pm0.01$ & $0.90\pm0.01$ \\
        20  & 1.22 & 1.23 & 1.21 & 0.56 & 0.56 & 0.57 & $1.22\pm0.01$ & $0.56\pm0.01$ \\
        30  & 1.08 & 1.07 & 1.04 & 0.36 & 0.35 & 0.35 & $1.06\pm0.02$ & $0.35\pm0.01$ \\
        40  & 1.11 & 1.09 & 1.06 & 0.27 & 0.28 & 0.28 & $1.09\pm0.03$ & $0.28\pm0.01$ \\
        45  & 1.12 & 1.12 & 1.06 & 0.24 & 0.25 & 0.26 & $1.10\pm0.04$ & $0.25\pm0.01$ \\
        50  & 1.26 & 1.25 & 1.17 & 0.26 & 0.26 & 0.28 & $1.23\pm0.05$ & $0.27\pm0.01$ \\
        60  & 1.53 & 1.55 & 1.52 & 0.36 & 0.35 & 0.37 & $1.53\pm0.02$ & $0.36\pm0.01$ \\
        70  & 1.80 & 1.84 & 1.83 & 0.55 & 0.55 & 0.59 & $1.82\pm0.02$ & $0.56\pm0.02$ \\
        80  & 1.85 & 1.91 & 1.93 & 0.95 & 0.93 & 0.91 & $1.90\pm0.04$ & $0.93\pm0.02$ \\
        90  & 1.59 & 1.82 & 1.82 & 1.42 & 1.46 & 1.46 & $1.74\pm0.13$ & $1.45\pm0.02$ \\
        100 & 2.42 & 2.44 & 2.40 & 1.53 & 1.53 & 1.57 & $2.42\pm0.02$ & $1.54\pm0.02$ \\
        110 & 2.99 & 2.95 & 2.97 & 1.33 & 1.32 & 1.31 & $2.97\pm0.02$ & $1.32\pm0.01$ \\
        120 & 3.53 & 3.70 & 3.56 & 1.16 & 1.12 & 1.09 & $3.60\pm0.09$ & $1.12\pm0.04$ \\
        130 & 4.15 & 3.91 & 4.11 & 1.08 & 1.01 & 1.01 & $4.06\pm0.13$ & $1.03\pm0.04$ \\
        140 & 3.92 & 3.74 & 4.03 & 1.07 & 0.98 & 0.98 & $3.90\pm0.15$ & $1.01\pm0.05$ \\
        150 & 3.47 & 3.74 & 3.49 & 1.10 & 1.15 & 1.08 & $3.57\pm0.15$ & $1.11\pm0.04$ \\
        160 & 3.06 & 2.82 & 3.07 & 1.41 & 1.28 & 1.33 & $2.98\pm0.14$ & $1.34\pm0.07$ \\
        170 & 2.22 & 2.41 & 2.16 & 1.51 & 1.45 & 1.30 & $2.26\pm0.13$ & $1.42\pm0.11$ \\
        180 & 1.61 & 1.50 & 1.65 & 1.31 & 1.44 & 1.33 & $1.59\pm0.08$ & $1.36\pm0.07$ \\
        \bottomrule
    \end{tblr}
\end{table}
The measured maximum and minimum voltages corresponding to the polarization angle $\phi$ in the range \SI{0}{\degree}–\SI{180}{\degree} listed in \autoref{tab:contrast}.
Each angle $\phi$ has a reading uncertainty of $\pm\SI{2}{\degree}$, and each individual voltage measurement has an instrumental uncertainty of $\pm\SI{0.01}{\volt}$.
The uncertainties reported for $\bar{U}_{\text{max/min}}$ are the sample standard deviations from the three measurements.
In addition, a background noise level of \SI{0}{\volt} was measured, indicating that no offset needed to be subtracted from the recorded voltages.
Therefore, the values listed in \autoref{tab:contrast} were used directly in the contrast calculations.\\
Since the photodiode output voltage is proportional to the light intensity $I \sim U$, the mean voltages $U_{\text{max/min}}$ can be used directly in place of intensity values for determining the interferometer contrast. 
The calculated contrast using the equation \ref{eqn:contrast} are plotted in \autoref{fig:contrast} as well as the theoretical curve described by the equation \ref{eqn:contrast_Sagnac}.
\begin{figure}[H]
  \centering
  \includegraphics[width=0.9\linewidth]{build/contrast.pdf}
  \caption{Interferometer contrast as a function of polarization angle $\phi$, showing measured values and the corresponding theoretical curve.}
  \label{fig:contrast}
\end{figure}
Based on the measured values and the theoretical contrast curve, a polarization angle of \SI{135}{\degree} was chosen for the subsequent measurements. This setting yields a contrast close to the theoretical maximum. More importantly, this setting provides a significantly higher signal amplitude compared to \SI{45}{\degree} which has similar contrast but a much lower voltage swing. The increased signal amplitude at \SI{135}{\degree} improves the signal-to-noise ratio, allowing for more precise detection of interference fringes and phase shifts.

\subsection{Refraction index of glass}
\label{sec:glass_index}
To determine the refractive index $n$ of the glass plates, we measured the number of interference fringes $M$ that appeared while rotating the mounted double-glass holder by an additional angle of $\Theta = \SI{10 \pm 2}{\degree}$. 
Since both glass plates have an initial tilt of $\Theta_0 = \pm\SI{10}{\degree}$, the equation \ref{eqn:delta_glass_vontheta} has to be modified by using equation \ref{eqn:delta_diff}. When using the relation \ref{eqn:M}, this leads to
\begin{align}
    M &= \frac{d}{\lambda_{\text{vac}}} \cdot \frac{n - 1}{2n} \left[ (\Theta_0 + \Theta)^2 - (\Theta_0 - \Theta)^2 \right] \nonumber \\
      &= \frac{d}{\lambda_{\text{vac}}} \cdot \frac{n - 1}{2n} (4 \Theta_0 \Theta) \nonumber \\
\Leftrightarrow\quad 
    n &= \frac{2d \Theta_0 \Theta}{2d \Theta_0 \Theta - M \lambda_{\text{vac}}} \,. \label{eqn:fraction_glass}
\end{align}
Using the equation \ref{eqn:fraction_glass} with $d=\SI{1}{\milli\meter}$ and the arithmetic mean of the measured interference maxima from \autoref{tab:glass_counts}, the refractive index of the glass plates is calculated. This results in a refractive index of 
$$
    n_{\text{glass, exp}} = \SI{1.49\pm0.21}\,.
$$
\begin{table}[h]
    \centering
    \caption{Measured numbers $M$ of interference maxima while rotating a glass plate by $\Theta=\SI{10\pm2}{\degree}$ and the resulting arithmetic mean $\bar{M}$.}
    \label{tab:glass_counts}
    \begin{tblr}{colspec= c|c c c c c c c c c c || c}
        \toprule
        $M$ & 28 & 27 & 33 & 33 & 33 & 33 & 33 & 33 & 32 & 32 & $\bar{M} = 31.7\pm2.1$\\
        \bottomrule
    \end{tblr}
\end{table}

\subsection{Refraction index of air}
\label{sec:air}
In this part of the experiment, the change in the refractive index of air with respect to pressure was investigated using an interferometric method. The numbers of observed interference fringes $M$ was recorded for various pressures, with an instrumental uncertainty in pressure of \SI{\pm1}{\milli\bar}. 
For each pressure, the arithmetic mean $\bar{M}$ and the corresponding refractive index $n_{\text{air}}$ were calculated. The results are summarized in \autoref{tab:air}.
\begin{table}[t]
    \centering
    \caption{Measured numbers $M$ of interference maxima at different air pressure $p$ with the resulting arithmetic mean $\bar{M}$ and calculated refraction index $n$. The three measurements were conducted at the temperatures $T_1=\SI{21.5}{\celsius},\,\,T_2=\SI{21.6}{\celsius}$ and $T_3=\SI{21.7}{\celsius}$.}
    \label{tab:air}
    \begin{tblr}{colspec= c|c c c ||c c}
        \toprule
        $p\,[\unit{\milli\bar}]$ & $M_1$ & $M_2$ & $M_3$ & $\bar{M}$ & $n_{\text{air}}(p)$\\
        \midrule
        7   &0 &  0  &   0 & $0.00\pm0.00$  & $1.000000000 \pm 0.00 \cdot 10^{0}$ \\
        50  &2 &  2  &   2 & $2.00\pm0.00$  & $1.000012660 \pm 1.27 \cdot 10^{-8}$ \\
        100 &4 &  4  &   4 & $4.00\pm0.00$  & $1.000025320 \pm 2.53 \cdot 10^{-8}$ \\
        150 &6 &  7  &   6 & $6.33\pm0.47$  & $1.000040090 \pm 2.98 \cdot 10^{-6}$ \\
        200 &8 &  9  &   8 & $8.33\pm0.47$  & $1.000052749 \pm 2.98 \cdot 10^{-6}$ \\   
        250 &11&  11 &  11 & $11.00\pm0.00$ & $1.000069629 \pm 6.96 \cdot 10^{-8}$ \\
        300 &13&  13 &  13 & $13.00\pm0.00$ & $1.000082289 \pm 8.23 \cdot 10^{-8}$ \\  
        350 &15&  15 &  15 & $15.00\pm0.00$ & $1.000094949 \pm 9.49 \cdot 10^{-8}$ \\    
        400 &17&  17 &  17 & $17.00\pm0.00$ & $1.000107608 \pm 1.08 \cdot 10^{-7}$ \\     
        450 &19&  19 &  19 & $19.00\pm0.00$ & $1.000120268 \pm 1.20 \cdot 10^{-7}$ \\     
        500 &21&  21 &  21 & $21.00\pm0.00$ & $1.000132928 \pm 1.33 \cdot 10^{-7}$ \\      
        550 &23&  23 &  23 & $23.00\pm0.00$ & $1.000145588 \pm 1.46 \cdot 10^{-7}$ \\      
        600 &25&  25 &  25 & $25.00\pm0.00$ & $1.000158248 \pm 1.58 \cdot 10^{-7}$ \\     
        650 &27&  28 &  27 & $27.33\pm0.47$ & $1.000173017 \pm 2.99 \cdot 10^{-6}$ \\    
        700 &29&  30 &  29 & $29.33\pm0.47$ & $1.000185677 \pm 2.99 \cdot 10^{-6}$ \\    
        750 &32&  32 &  32 & $32.00\pm0.00$ & $1.000202557 \pm 2.03 \cdot 10^{-7}$ \\         
        800 &34&  34 &  34 & $34.00\pm0.00$ & $1.000215217 \pm 2.15 \cdot 10^{-7}$ \\         
        850 &36&  36 &  36 & $36.00\pm0.00$ & $1.000227876 \pm 2.28 \cdot 10^{-7}$ \\        
        900 &38&  38 &  38 & $38.00\pm0.00$ & $1.000240536 \pm 2.41 \cdot 10^{-7}$ \\        
        950 &40&  40 &  40 & $40.00\pm0.00$ & $1.000253196 \pm 2.53 \cdot 10^{-7}$ \\        
        997 &42&  42 &  42 & $42.00\pm0.00$ & $1.000265856 \pm 2.66 \cdot 10^{-7}$ \\     
        \bottomrule
    \end{tblr}
\end{table}
The calculation of the refractive index was done by using the equations \ref{eqn:delta_gas} and \ref{eqn:M} and the relation
$$
    \symup{\Delta}n = n(p)-n(p_0)\,,
$$
where $n(p_0)\approx 1$ is the refractive index at standard atmospheric pressure. This leads to the expression
\begin{equation}
    n(p)= 1 + \frac{M\cdot \lambda_{\text{vac}}}{L}\,,\label{eqn:refraction_air}
\end{equation}
with $\lambda_{\text{vac}} = \SI{632.990}{\nano\meter}$ and $L=\SI{100.0\pm0.1}{\milli\meter}$ representing the vacuum wavelength of the laser and the effective path length. All values of $ n_{\text{air}}(p)$ are  listed in \autoref{tab:air} were obtained using this formula.
%  Calculating the mean of these refractive indices yields 
% $$
%     \bar{n}_{\text{air, 1}} = 1.00013363\pm0.00000031\,.
% $$
\subsubsection{Standard atmosphere}
\label{sec:LorentzLorenzLaw}
In order to determine the refrective index of air at standard atmosphere, the Lorentz-Lorenz Law \ref{eqn:Lorentz_Lorenz} is used.
Under standard conditions, the refractive index of air is very close to 1 (specifically, $n_{\text{air, theo}}=1.00027653$ at $\lambda=\SI{632.990}{\nano\meter}$ \cite{refractiveIndex}). Given how close $n$ is to $1$, it is physically reasonable to approximate the left-hand side of the Lorentz-Lorenz equation using a Taylor expansion around n=1: This leads to 
\begin{align}
    \frac{n^2 -1}{n^2 +2} &\approx n-1  + \mathcal{O} (n^2)\nonumber\\
    &\Rightarrow n-1 = \frac{\alpha}{3 \varepsilon _0 k_B} \frac{p}{T} \nonumber\\
    &\Leftrightarrow n = 1  + \underbrace{\frac{\alpha}{3 \varepsilon _0 k_B}}_{=:\beta} \frac{p}{T} \label{eqn:LLLair}\,.
\end{align}
The different measurements listed in \autoref{tab:air} were taken under slightly fluctuating temperatures. Therefore, the mean temperature is calculted as $\bar{T}=\SI{21.60\pm0.08}{\celsius}$ and used to plot the calculated values of $\bar{M}$ against $\sfrac{p}{\bar{T}}$, as shown in \autoref{fig:fit}.
To model the relationship, the equations \ref{eqn:refraction_air} and \ref{eqn:LLLair} are set equal, yielding a linear relation
\begin{align}
    M &= \frac{L \beta}{\lambda}\frac{p}{T} \nonumber\\
    \hat{M} &= a \frac{p}{T}+b\,. \label{eqn:linearfit}
\end{align}
This linear equation is fitted to the measured data, resulting in the following parameters
\begin{align*}
    a &= \SI{12.46\pm0.05}{\kelvin\per\milli\bar}\\
    b &= -0.03\pm0.09\,.
\end{align*}
\begin{figure}[H]
    \centering
    \includegraphics[width=0.9\linewidth]{build/fit.pdf}
    \caption{Linear relation between the number of fringes $M$ and $\sfrac{p}{T}$ and a linear fit.}
    \label{fig:fit}
\end{figure}
Combining the equations \ref{eqn:refraction_air} and \ref{eqn:linearfit} with the calculated parameters as well as $\lambda_{\text{vac}},\,\,p_0 =\SI{1013}{\milli\bar}$ and $T_0 = \SI{15}{\degree}$ at normal condition to
$$
n(p_0, T_0) = \frac{\lambda_{\text{vac}}}{L} \hat{M} +1$$ 
leads to the refractive index of air $n_{\text{air, exp}} = 1.0002770\pm0.0000013$.