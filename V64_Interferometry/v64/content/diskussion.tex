\section{Discussion}
\label{sec:Discussion}
The relative deviation is calculated using the following equation
\begin{equation*}
    \symup{\Delta} x_{\text{rel}} = \left| \frac{x_{\text{exp}} - x_{{\text{theo}}}}{x_{{\text{theo}}}}\right|\,.
\end{equation*}
The theoretical value for the refractive index of air is given as $n_{\text{air, theo}}=1.00027653$, while the refractive index of glass depends on the specific material and typically lies within the range $n_{\text{glas, theo}}\in [1.4570, 1.8449]$ at a wavelength of $\lambda = \SI{632.990}{\nano\meter}$. 
In \autoref{sec:Analysis}, the experimental value for the refractive index of air was determined to be $n_{\text{air, exp}} = 1.0002770\pm0.0000013$, resulting in a relative deviation of $\symup{\Delta}n_{\text{air}} = 0.00005015\,\%$. 
The experimentally determined refractive index of glass was calculated to be $n_{\text{glass, exp}} = 1.49\pm0.21$ which lies well within the expected theoretical range. This confirms that the measured value corresponds to a physically plausible glass material.