\section{Discussion}
\label{sec:Discussion}
The relative deviation is calculated using the following equation
\begin{equation*}
    \symup{\Delta} x_{\text{rel}} = \left| \frac{x_{\text{exp}} - x_{{\text{theo}}}}{x_{{\text{theo}}}}\right|\,.
\end{equation*}
\subsection{Contrast}
The measured contrast was lower than the theoretically expected contrast throughout the range. The reason for this could be 
that the theoretical contrast was calculated for an ideal system with the beams exactly aligned. This cannot be achieved in 
a real experiment. Furthermore, the interfering beams had quite a few fringes when observed. This shows that the beams 
were not entirely parallel and therefore the contrast is lower than ideal. 

\subsection{Refractive indices}
The theoretical value for the refractive index of air is given as $n_{\text{air, theo}}=1.00027653$. 
In \autoref{sec:Analysis}, the experimental value for the refractive index of air was determined to be 
$n_{\text{air, exp}} = 1.0002770\pm0.0000013$, resulting in a relative deviation of $\symup{\Delta}n_{\text{air}} = 0.00005015\,\%$. 
The refractive index of 
glass depends on the specific material and typically lies within the range $n_{\text{glass, theo}}\in [1.4570, 1.8449]$ at a 
wavelength of $\lambda = \SI{632.990}{\nano\meter}$. 
The experimentally determined refractive index of glass was calculated to be $n_{\text{glass, exp}} = 1.49\pm0.21$ which lies 
well within the expected theoretical range. 
This confirms that the measured value corresponds to a physically plausible glass material.
A reason for deviation of the refractive index of glass could be that it was difficult to accurately read the degree of rotation 
of the glass holder. 
Since the degree of rotation is crucial for the calculation of the refractive index, any uncertainty in its measurement could significantly impact the result and thus be a source of deviation.
An explanation for the minor deviations of both refractive indices could be that a Plexiglas hood was placed over the 
experimental setup to minimize 
variation of air density.