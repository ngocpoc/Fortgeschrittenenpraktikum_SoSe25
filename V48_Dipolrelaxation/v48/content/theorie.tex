\section{Zielsetzung}
\label{sec:Zielsetzung}
Das Ziel des Versuchs ist die Messung der Relaxationszeit von Dipolen in dotierten Ionenkristallen. Durch den Versuch werden die charakteristische Relaxationszeit und die Aktivierungsernergie bestimmt. 
\section{Theorie}
\label{sec:Theorie}
Die theoretischen Ausführungen basieren auf \cite{Fuller}. 
\subsection{Ionenkristalle} 
\label{Ionenkristalle} 
Ionenkristalle sind Festkörper, die aus Anionen und Kationen zusammengesetzt sind. Beispiele für Ionenkristalle sind Kaliumbromid bestehend aus einfach positiv geladenem Kalium und einfach negativ geladenen Brom und Cäsiumiodid bestehend aus einfach positiv geladenem Cäsium und einfach negativ geladenen Iod. Kaliumbromid bildet eine kubisch-flächenzentrierte Struktur aus \cite{NaCl_Struktur} \cite{fcc_Struktur}. Der ideale Kaliumbromid bzw. Cäsiumiodid Kristall ist unendlich groß und ist elektrisch neutral. Durch Störstellen oder Dotierung kann eine Dipolstruktur im Kristall erzeugt werden. Da die Dotierung bzw. die Störstellen statistische verteilt sind, ist in diesem Zustand kein Gesamtdipolmoment des Kristalls messbar. 

\label{Relaxationszeit und Aktivierungsenergie}
Die Relaxationszeit $\tau$ und die Aktivierungsenergie $W$ sind wichtige Parameter zur Charakterisierung von Dipolen. Die Relaxationszeit bezeichnet die Zeit, die ein Dipol benötigt, um nach Ausrichtung z.B. durch ein extern angelegtes elektrisches Feld wieder in den unpolarizierten Grundzustand überzugehen. Die Aktivierungsenergie $W$ ist die Energie, die benötigt wird, um die Polarisation des Dipols zu erreichen. $W$ ist temperaturabhängig. Außerdem folgt die Energieverteilung einer Boltzmannverteilung. Es gilt 
\begin{equation}
    E \propto \exp{\left( \frac{W}{k_\text{B} T}\right)} \, .
\end{equation}
$k_\text{B}$ ist die Bolzmannkonstante und $T$ die Temperatur. Darauß ergibt sich für die die Relaxationszeit 
\begin{equation}
    \tau(T) = \tau_0 \cdot \exp{\left( \frac{W}{k_\text{B} T}\right)} \, . \label{eqn:tau_exp}
\end{equation}
$\tau_0 = \tau(\infty)$ ist die charakteristische Relaxationszeit. 

\label{Ionen-Thermostrom-Methode}
Zur Bestimmung der Relaxationszeit und Aktivierungsenergie wird die Ionen-Thermostrom-Methode verwendet. Bei dieser Methode wird bei einer Temperatur $T_1$ ein homogenes elektrisches Feld angelegt für die Zeit $t_1 \gg \tau {\left( T_1 \right)}$. Nach dieser Zeit sollte die Probe vollständig polarisiert sein. Anschließend wird die Probe auf die Temperatur $T_0$ heruntergekühlt. Danach wird das elektrische Feld abgeschaltet und die Probe mit konstanter Heizrate $b = \frac{\symup{d}T}{\symup{d}t}$ erhöht. Bei steigender Temperatur relaxieren die Dipole allmählich in ihren Grundzustand. Dadurch entsteht ein temperaturabhängiger Depolarisationsstrom $I$, der durch 
\begin{equation}
    I(T) = \textcolor{PineGreen}{\frac{N p² E}{3 k_\text{B} T_1}} \textcolor{RoyalPurple}{ \frac{1}{\tau_0} \exp{\left(- \frac{W}{k_\text{B} T}\right)}} \textcolor{RedViolet}{\exp{\left( -(b \tau_0)^{-1}  \int_{T_0}^{T} \exp{\left(-\frac{W}{k_\text{B} T'}\right)}\,\symup{d}T'\right)}}
    \label{eqn:grosse_I_Formel}
\end{equation}
beschrieben wird. Der grüne Teil beschreibt dabei die durchschnittliche Polarisation pro Dipol, der blaue Teil beschreibt die Relaxationsrate und der rote Teil die Anzahl der für den Strom relevanter Dipole. $p$ bezeichnet dabei das Dipolmoment eines einzelnen Dipols und $N$ ist die Anzahl der Dipole pro Einheitsvolumen. Das Maximum der Kurve ist unabhängig von der ursprünglichen Temperatur $T_1$ auf die die Probe aufgeheizt wurde und ist bei der Temperatur  
\begin{equation}
    T_{\text{max}} = \sqrt{\frac{b W \tau\left(T_{\text{max}}\right)}{k_\text{B}}}
    \label{eqn:T_max}
\end{equation} 
gegeben. Bei tiefen Temperaturen, kann der Verlauf der Kurve durch 
\begin{equation}
    \ln(I(T)) = \text{const} - \frac{W}{k_\text{B} T}
    \label{eqn:approx_kalte_Temperaturen}
\end{equation} 
approximiert werden.
Zur Bestimmung der Aktivierungsenergie $W$ kann der Zusammenhang 
\begin{equation}
    \ln(\tau(T)) = \ln(\tau_0) + \frac{W}{k_\text{B} T} = \ln \left(\frac{1}{I(T)} \int_{T}^{\infty}I(T')\,\symup{d}T' \right)
    \label{eqn:fuer_W_Bestimmung}
\end{equation} 
verwendet werden. 
%




%Nach Polarisierung der Dipole durch ein elektrisches Feld entsteht durch die in den Grundzustand übergehende Dipole ein Depolarisationsstrom, der temperaturabhängig ist. Dieser kann durch Verwendung der Debye-Näherung und der Annahme, dass die einzelnen Dipole nicht miteinander interferieren, hergeleitet werden. Unter Verwendung dieser Näherungen kann die durchschnittliche Polarisation $P$ durch 
%\begin{equation}
%    P = \frac{p² E}{3 k_\text{B} T}
%\end{equation}
%beschrieben werden. $p$ bezeichnet dabei das Dipolmoment eines einzelnen Dipols. Außerdem ist die Anzahl der Dipole, die zum Depolarisationsstrom beitragen, temperaturabhängig und folgt der typischen Relaxationsgleichung 
%\begin{equation}
%    \frac{\symup{d}N}{\symup{d}t} = - \frac{N}{\tau(T)} \, .
%\end{equation} 
%$N$ ist dabei die Anzahl der Dipole pro Einheitsvolumen
%
%\cite{anleitungV48} 
%\cite{IonicThermo}
%\cite{Irradiation}
%\cite{Bunsengesellschaft}
%\cite{Fuller}
% \subsection{Vorbereitungsaufgaben}
% \label{sec:Vorbereitungsaufgaben}
%https://www.korth.de/material/detail/Kaliumbromid für NaCl Struktur
%https://www.chemie.de/lexikon/Natriumchlorid-Struktur.html für NaCl = fcc Gitter 