\section{Diskussion}
\label{sec:Diskussion}
Die Abweichung $\Delta x_{\text{rel}}$ der experimentell bestimmten Werte von den Theoriewerten wird mithilfe von 
\begin{equation*}
    \Delta x_{\text{rel}} = \left| \frac{x_{\text{exp}} - x_{\text{theo}}}{x_{\text{theo}}}  \right|
\end{equation*}
berechnet. Die theoretische charakteritische Relaxationszeit ist $\tau_{\text{0,theo}} = 4 \cdot 10^{-14} \, \unit{\second}$ \cite{Irradiation}. Die Abweichungen zu den experimentell bestimmten Werten $\tau_{0,1} = \left(4\pm4\right)\cdot 10^{-17}\,\si{\second}$ und $\tau_{0,2} = \left(5\pm9\right)\cdot 10^{-16}\,\si{\second}$ sind $\Delta \tau_{1} = 99,90 \, \%$ und $\Delta \tau_{2} = 98,75 \, \%$.\\
Die theoretische Aktivierungsenergie ist $W_{\text{theo}} = 0,66 \, \unit{\electronvolt}$ \cite{Irradiation}. Die Abweichungen zu den experimentell bestimmten Werten $𝑊_{1} = (1, 538 \pm 0, 032) \cdot 10^{−19} \, \unit{\electronvolt}$ und
$𝑊_2 = (1, 45 \pm 0, 07) \cdot 10^{−19} \, \unit{\electronvolt}$ sind $\Delta W_{1} = \text{Krise} \, \%$ und $\Delta W_{2} = \text{Krise} \, \%$.

\subsection{Mögliche Fehlerquellen}
Eine mögliche Fehlerquelle könnte die Probe selbst darstellen, da diese hygroskopisch ist und nicht gewechselt wird. Das angesammelte Wasser könnte die Messergebnisse verfälschen, da es ein starkes Dipolmoment hat.  Außerdem sind die elektronischen Felder sehr sensibel auf Bewegungen. Das stellt eine Problematik beim Ablesen dar, da die Messwerte nur von einigen Metern entfernung aufgenommen werden sollten, das das Ablesen aber schwieriger macht. Außerdem ist das Experiment in der Nähe des Eingangs aufgebaut. Das elektrische Feld könnte durch die Bewegung der vorbeilaufenden Menschen gestört worden sein. 
Zudem konnte die Zeit zwischen dem Aufnehmen der Messwerte nicht exakt konstant gehalten werden, da das Ablesen verschieden lange dauerte. Außerdem konnte die Heizrate nicht ganz konstant gehalten werden, da das System verzögert auf Nachregeln von Heizspannung reagiert. All diese Fakoren können den sehr kleinen gemessenen Strom empfindlich stören. 