\section{Auswertung}
\label{sec:Auswertung}

\subsection{Bestimmung der Heizraten}
\begin{figure}
  \centering
  \includegraphics[width=0.8\linewidth]{build/Heizraten.pdf}
  \caption{Gemessene Temperatur gegen die Zeit der beiden Messreihen und eine lineare Regression zur Bestimmung der Heizraten.}
  \label{fig:Heizraten}
\end{figure}
\FloatBarrier
\subsection{Bestimmung des Untergrunds und bereinigte Daten}
\begin{figure}
    \centering
    \includegraphics[width=0.8\linewidth]{build/ersterStrom.pdf}
    \caption{Gemessene Daten der ersten Messreihe mit exponentiellen Fit für den Untergrund sowie die gekennzeichneten exkludierten Messdaten.}
    \label{fig:Untergrund1}
\end{figure}

\begin{figure}
    \centering
    \includegraphics[width=0.8\linewidth]{build/zweiterStromVor.pdf}
    \caption{Gemessene Daten der zweiten Messreihe mit exponentiellen Fit für den Untergrund sowie die gekennzeichneten exkludierten Messdaten.}
    \label{fig:Untergrund2}
\end{figure}

\begin{figure}
    \centering
    \includegraphics[width=0.8\linewidth]{build/ersterUntergrundfrei.pdf}
    \caption{Bereinigte Messdaten der ersten Messreihe mit Kennzeichnung der verwendeten Messdaten für die zwei Methoden zur Bestimmung der Aktivierungsenergie.}
    \label{fig:Untergrundfrei1}
\end{figure}

\begin{figure}
    \centering
    \includegraphics[width=0.8\linewidth]{build/zweiterUntergrundfrei.pdf}
    \caption{Bereinigte Messdaten der zweiten Messreihe mit Kennzeichnung der verwendeten Messdaten für die zwei Methoden zur Bestimmung der Aktivierungsenergie.}
    \label{fig:Untergrundfrei2}
\end{figure}

\FloatBarrier
\subsection{Aktivierungsenergie aus dem Polarisationsansatz}
\begin{figure}
    \centering
    \includegraphics[width=0.8\linewidth]{build/polarisation1.pdf}
    \caption{Messdaten sowie linearer Fit der ersten Messreihe zur Bestimmung der Aktivierungsenergie mithilfe des Polarisationsansatzes.}
    \label{fig:Polarisation1}
\end{figure}

\begin{figure}
    \centering
    \includegraphics[width=0.8\linewidth]{build/polarisation2.pdf}
    \caption{Messdaten sowie linearer Fit der zweiten Messreihe zur Bestimmung der Aktivierungsenergie mithilfe des Polarisationsansatzes.}
    \label{fig:Polarisation1}
\end{figure}

\FloatBarrier  
\subsection{Aktivierungsenergie aus dem Stromdichtenansatz}
\begin{figure}
    \centering
    \includegraphics[width=0.8\linewidth]{build/stromdichte1.pdf}
    \caption{Messdaten sowie linearer Fit der ersten Messreihe zur Bestimmung der Aktivierungsenergie mithilfe des Stromdichtenansatzes.}
    \label{fig:Stromdichte1}
\end{figure}

\begin{figure}
    \centering
    \includegraphics[width=0.8\linewidth]{build/stromdichte2.pdf}
    \caption{Messdaten sowie linearer Fit der zweiten Messreihe zur Bestimmung der Aktivierungsenergie mithilfe des Stromdichtenansatzes.}
    \label{fig:Stromdichte2}
\end{figure}

\FloatBarrier
\subsection{Bestimmung der Relaxationszeit}
\begin{figure}
    \centering
    \includegraphics[width=0.8\linewidth]{build/relaxation.pdf}
    \caption{Temperaturabhängige Relaxationszeit $\tau$ der beiden Messreihen.}
    \label{fig:Relaxation}
\end{figure}


% Heizraten:
% Erste Messung
% (Heizrate) m1: 0.855+/-0.013 K/min
% b1: 210.1+/-0.8 K
% Zweite Messung
% (Heizrate) m2: 2.650+/-0.032 K/min
% b2: 195.1+/-0.7 K


% \begin{figure}
%   \centering
%   \includegraphics{plot.pdf}
%   \caption{Plot.}
%   \label{fig:plot}
% \end{figure}

%Siehe \autoref{fig:plot}!